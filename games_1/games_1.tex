% arara: xelatex
\documentclass[12pt]{article}

\usepackage{physics}


\usepackage{tikz} % картинки в tikz
\usepackage{microtype} % свешивание пунктуации

\usepackage{array} % для столбцов фиксированной ширины

\usepackage{indentfirst} % отступ в первом параграфе

\usepackage{sectsty} % для центрирования названий частей
\allsectionsfont{\centering}

\usepackage{amsmath, amsfonts, amssymb} % куча стандартных математических плюшек

\usepackage{comment}

\usepackage[top=2cm, left=1.2cm, right=1.2cm, bottom=2cm]{geometry} % размер текста на странице

\usepackage{lastpage} % чтобы узнать номер последней страницы

\usepackage{enumitem} % дополнительные плюшки для списков
%  например \begin{enumerate}[resume] позволяет продолжить нумерацию в новом списке
\usepackage{caption}

\usepackage{url} % to use \url{link to web}

\usepackage{fancyhdr} % весёлые колонтитулы
\pagestyle{fancy}
\lhead{Задачки для тигров-1}
\chead{}
\rhead{}
\lfoot{}
\cfoot{}
\rfoot{\thepage/\pageref{LastPage}}
\renewcommand{\headrulewidth}{0.4pt}
\renewcommand{\footrulewidth}{0.4pt}

\usepackage{tcolorbox} % рамочки!

\usepackage{todonotes} % для вставки в документ заметок о том, что осталось сделать
% \todo{Здесь надо коэффициенты исправить}
% \missingfigure{Здесь будет Последний день Помпеи}
% \listoftodos - печатает все поставленные \todo'шки


% более красивые таблицы
\usepackage{booktabs}
% заповеди из докупентации:
% 1. Не используйте вертикальные линни
% 2. Не используйте двойные линии
% 3. Единицы измерения - в шапку таблицы
% 4. Не сокращайте .1 вместо 0.1
% 5. Повторяющееся значение повторяйте, а не говорите "то же"



\usepackage{fontspec}
\usepackage{polyglossia}

\setmainlanguage{russian}
\setotherlanguages{english}

% download "Linux Libertine" fonts:
% http://www.linuxlibertine.org/index.php?id=91&L=1
\setmainfont{Linux Libertine O} % or Helvetica, Arial, Cambria
% why do we need \newfontfamily:
% http://tex.stackexchange.com/questions/91507/
\newfontfamily{\cyrillicfonttt}{Linux Libertine O}

\AddEnumerateCounter{\asbuk}{\russian@alph}{щ} % для списков с русскими буквами
\setlist[enumerate, 2]{label=\asbuk*),ref=\asbuk*}

%% эконометрические сокращения
\DeclareMathOperator{\Cov}{\mathbb{C}ov}
\DeclareMathOperator{\Corr}{\mathbb{C}orr}
\DeclareMathOperator{\Var}{\mathbb{V}ar}

\let\P\relax
\DeclareMathOperator{\P}{\mathbb{P}}

\DeclareMathOperator{\E}{\mathbb{E}}
% \DeclareMathOperator{\tr}{trace}
\DeclareMathOperator{\card}{card}
\DeclareMathOperator{\plim}{plim}
\DeclareMathOperator{\pCorr}{\mathrm{p}\mathbb{C}\mathrm{orr}}


\newcommand \hb{\hat{\beta}}
\newcommand \hs{\hat{\sigma}}
\newcommand \htheta{\hat{\theta}}
\newcommand \s{\sigma}
\newcommand \hy{\hat{y}}
\newcommand \hY{\hat{Y}}
\newcommand \e{\varepsilon}
\newcommand \he{\hat{\e}}
\newcommand \z{z}
\newcommand \hVar{\widehat{\Var}}
\newcommand \hCorr{\widehat{\Corr}}
\newcommand \hCov{\widehat{\Cov}}
\newcommand \cN{\mathcal{N}}
\newcommand \RR{\mathbb{R}}
\newcommand \NN{\mathbb{N}}
\newcommand{\cF}{\mathcal{F}}
\newcommand{\cH}{\mathcal{H}}


\begin{document}


\begin{enumerate}

   \item Кортес с бандой головорезов высадился на берегу. Кортес выбирает, нападать ли на деревушку
   или нет. Местная деревушка может либо сразу перейти в подчинение Кортеса, либо принять бой.
   Если деревушка примет бой, то выбор появится у Кортеса: либо драться до победного конца, либо
   после первых потерь бежать на кораблях обратно. Ценность деревушки для Кортеса — одна еди-
   ница, ценность собственных головорезов — 2 единицы. Если Кортес будет драться до конца, то
   деревушка будет взята, но большинство головорезов погибнет в бою. Для жителей деревушки —
   главное остаться в живых, но и сохранить при этом независимость, конечно, желательно.
   
   \begin{enumerate}
    \item Нарисуйте дерево игры и найдите исход методом обратной индукции.
    \item Что изменится, если Кортес сжигает корабли?
   \end{enumerate}
   
   \item На острове живут 99 тигров и одна вкусная волшебная антилопа.
   Если тигр съест волшебную антилопу, то он сам превратится в волшебную антилопу. 
   Мясо волшебной антилопы настолько вкусно, что любой тигр готов ради его вкуса на превращение в антилопу. 
   Но ни один тигр не готов полностью расстаться с жизнью ради мяса антилопы. Тигры
   охотятся только в одиночку.
   
   Что будет происходить на этом острове?

   
   \item В 1612 г. в Лионе появилась книга поэта и математика Баше де Мезирьяка (Claude Gaspar Bachet
   de Méziriac) «Занимательные и приятные числовые задачи» (Problèmes plaisants et délectables qui
   se font par les nombres). В ней была предложена следующая игра. Двое по очереди называют числа
   от 1 до 10, выигрывает тот, кто первым доведет сумму до 100. 
   
   
   В чью пользу эта игра?
   
   Примечание: Баше де Мезирьяк перевел с греческого на латынь Арифметику Диофаната, 
   на полях которой Ферма сформулировал свою великую теорему.   

   \newpage

   \item Полный золота торговый корабль был захвачен $n\geq 3$ абсолютно рациональными пиратами.
   У пиратов есть строгая иерархия: капитан, первый помощник капитана, второй помощник и т.д.
   Пираты делят золото так: сначала капитан предлагает свой вариант дележа, затем пираты голо-
   суют за или против: если дележ одобрен более чем половиной пиратов (включая предложившего
   дележ), то он принимается, а если нет, то капитана убивают, и дележ предлагает первый помощ-
   ник\ldots
   
   Каждый пират хочет остаться в живых и получить побольше золота. При одинаковых выгодах
   для себя пират голосует за тот вариант, где в живых остается больше сотоварищей.
   
   Какой дележ будет реализован (предположим, что золото бесконечно делимо)?



   \item Есть три рулетки: на первой равновероятно выпадают числа 2, 4 и 9; на второй — числа 1, 6 и 8; на
   третьей — числа 3, 5 и 7. Сначала первый игрок выбирает рулетку себе, 
   затем второй игрок выбирает рулетку себе из двух оставшихся. После этого рулетки, выбранные игроками, запускаются, и
   случай определяет победителя. Победителем считается тот, чья рулетка покажет большее число.
   Победитель получает от проигравшего 100 рублей.

   Какова шансы на победу каждого игрока при правльной игре?

   \item Количество денег в волшебной шкатулке постоянно увеличивается! Время дискретно. В момент
   времени $t \in \{1, 2, 3, \ldots, 100\}$ там находится $2t$ рублей. Каждый из двух игроков решает, когда
   ему потребовать деньги. Тот кто потребует деньги первым — получает сумму полностью, тот, кто
   потребует вторым — не получает ничего. Если требования поступают одновременно, то игроки
   делят сумму в шкатулке поровну. Если никто не потребует деньги к моменту $t = 100$, то деньги
   сгорают.

   Как будут развиваться события при абсолютной рациональности игроков?

   \item Три дуэлянта решили стреляться из-за Прекрасной Дамы. 
   Они одновременно делают выстрел, каждый сам выбирает,
   в кого целиться. Если после первого выстрела в живых осталось больше одного человека, 
   то выжившие снова одновременно стреляют недруг в недруга. Труэль продолжается до тех пор, пока
   в живых не останется один человек или пока все не погибнут. Первый попадает с вероятностью
   $0.9$, второй — с вероятностью $0.5$, третий — с вероятностью $0.1$.
   
   \begin{enumerate}
      \item В кого кому следует стрелять, чтобы максимизировать вероятность своей победы?
      \item Найдите вероятность победы каждого игрока.
   \end{enumerate}
   

\end{enumerate}


\end{document}
